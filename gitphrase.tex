\documentclass[12pt]{article}
\usepackage[colorlinks=true,pagebackref,linkcolor=blue]{hyperref}
\usepackage{listings}
\usepackage{courier}
\lstset{
basicstyle=\footnotesize\ttfamily,
%numbers=left,
frame=bottomline,
framextopmargin=50pt,
}

\textwidth=7in
\textheight=9.5in
\topmargin=-1in
\headheight=0in
\headsep=.5in
\hoffset  -.85in

\pagestyle{empty}

\renewcommand{\thefootnote}{\fnsymbol{footnote}}


\begin{document}

\begin{center}
{\bf AMS 550.400 \quad MW 3:00 - 4:15 PM \quad  Room: Hodson 211}
\end{center}

While you may also do the same things with a GUI version of Git, 
here, we assume you will be using your terminal: 

\paragraph{Your very first step (or if you mess up or if you can't find
the folder)}

Assuming that you do not have a folder called ``550400.git" in your computer,
you can type in your command line:

\begin{lstlisting}
cd ~/
git clone http://cis.jhu.edu/~nhlee/550400.git
\end{lstlisting}
This creates a folder in your computer that is called 
``550400.git" and populates it afresh with the contents of 
the course folder.

\paragraph{Incremental update step}

Assuming that you did not mess up your git folder by,
for example, deleting or modifying things, do:

\begin{lstlisting}
cd ~/550400.git
git reset --hard HEAD
git pull origin master
\end{lstlisting}
This fetches new materials from the course git folder and 
merge them into your current git folder. 

\paragraph{Do \& Don't}

Use the course folder as if it is a bag of precious templates or references. 
In particular, unless you are comfortable with Git,
I would like to discourage (but not necessarily forbid) you 
from deleting, editing, adding things in the course folder and from 
changing the course folder name. 

Keep the folder relatively ``clean", and copy things from it to another folder.
Particularly, be very careful when dragging things from the course folder to 
another folder in your computer. 
In some computer, drag-and-drop means, in effect, cut-and-paste. 

\paragraph{Alternative} 

If you rather use the usual but inefficient routine to update your course
folder, feel free to visit: 
\begin{center}
    \url{http://cis.jhu.edu/~nhlee/550400.html/}
\end{center}
However, this alternative is not supported officially.  In particular, 
be warned that this html folder may contain files that you do not need to care
and sometimes, it may even contain a non-official version. In other words, use it at
your own risk. 
\end{document}
