\section{Textbook}

\subsection{Writing about Numbers}

\begin{frame}
    \frametitle{Outline}
    \tableofcontents[currentsection, currentsubsection]
\end{frame}
\begin{frame}
    \frametitle{Seven Basic Principles}
     \begin{enumerate}
         \item Set the context 
         \item Choose effective examples and analogies
         \item Choose vocabulary to suit your readers
         \item Decide whether to present \#s in text, tables, or figures
         \item Report and interpret \#s in the text
         \item Specify the direction \emph{and} size of an association between variables
         \item For many \#s, summarize overall pattern 
     \end{enumerate}
\end{frame}

\subsection{Math. Modeling}

\begin{frame}
    \frametitle{Outline}
    \tableofcontents[currentsection, currentsubsection]
\end{frame}

\begin{frame}
    \frametitle{Models and Reality: ``Disclaimer''}
    \begin{verse}
        Here we are concerned exclusively with mathematical models, that is,
        models that mimic reality by using the language of mathematics.
        Whenever we use ``model'' without a modifier, we mean ``mathematical
        model.''
    \end{verse}
\end{frame}

\begin{frame}
    \frametitle{Models and Reality}
    \begin{verse}
    What makes Mathematical models useful? 
    If we ``speak in mathematics,'', 
    \begin{itemize}
        \item We must formulate our ideas precisely and so are less likely to
            let implicit assumptions slip by,
        \item We have a concise ``language'' which encourages manipulation,
        \item We have a large number of potential theorems available,
        \item We have high speed computers available for carrying out calculations.
    \end{itemize}
    \end{verse}
\end{frame}

\begin{frame}
    \frametitle{Properties of Models}
    \begin{verse}
        A mathematical model is an abstract, simplified, mathematical
        construct related to a part of reality and created for a particular
        purpose.
    \end{verse} 
    \begin{verse} 
        Since a dozen different people are likely to come up with a
        dozen different definitions, don't take this one too seriously;
    \end{verse}
    \begin{verse}
        rather, think of it as a crude starting point around which to build
        your own understanding of mathematical modeling.
    \end{verse}
\end{frame}

\begin{frame}
    \frametitle{Properties of Models}
    \begin{verse}
    As far as a model is concerned, the world can be divided into three parts:
    \begin{enumerate}
        \item Things whose effects are neglected,
        \item Things that affect the model but whose behavior the model is not
            designed to study,
        \item Things the model is designed to study the behavior of. 
    \end{enumerate}
\end{verse}
\end{frame}

\begin{frame}
    \frametitle{Building a Model: ``Disclaimer''}
    \begin{verse}
        Model building involves imagination and skill. Giving rules for doing
        it is like listing rules for being an artist; at best this provides a
        framework around which to build skills and develop imagination. 
    \end{verse} 
    \begin{verse}
        It may
        be impossible to teach imagination. I won't try, but I hope this book
        provides an opportunity for your skills and imagination to grow. With
        these warnings, I present an outline of the modeling process.
    \end{verse}
\end{frame}


\begin{frame}
    \frametitle{Building a Model}
    \begin{verse}
        With these warnings, I present an outline of the modeling process.
    \begin{description}
        \item[1.] {Formulate a problem}
        \item[2.] {Outline the model}
        \item[3.] {Is it Useful?}
        \item[3.] {Test the model}
    \end{description}
    \end{verse}
\end{frame}

\begin{frame}
    \frametitle{Building a Model}
    \begin{verse}
        Some models may require no data. If a model makes the same prediction
        regardless of the data, we are not getting something for nothing because
        this prediction is based on the assumptions of the model. 
    \end{verse}
   
    \begin{verse}
        To some extent,
        the distinction between data and assumptions is artificial. In an extreme
        case, a model may be so specialized that its data are all built into the
        assumptions.
    \end{verse}
\end{frame}

\begin{frame}
    \frametitle{Building a Model}
    \begin{verse}
        The manager of a large commercial printing company 
        asks your advice on how many salespeople to employ.  
    \end{verse}
    \begin{verse}
        Qualitatively, more salespeople will increase sales overhead,
        while fewer salespeople may mean losing potential customers. 
    \end{verse}
    \begin{verse}
       Thus there should be some optimum number.
    \end{verse}
    
\end{frame}


\begin{frame}
    \frametitle{IMM Problem: ``Disclaimer''}
    \begin{verse}
        Some of the problems in this book lead you step by step through the
        development of a model and thus resemble the mathematics problems you have
        seen in other courses; 
    \end{verse} 
    \begin{verse} 
        however, many problems are closer to real life:
        They are vaguely stated, have multiple answers (models), or are open
        ended. 
    \end{verse}
    \begin{verse}
        I strongly recommend working in small groups on the problems to
        bring out various ideas and evaluate them critically.
    \end{verse}
\end{frame}

     
\begin{frame}
    \frametitle{Models and Reality}
    \begin{verse}
        The ultimate test of a model is how well it performs when 
        it is applied to the problem it was designed to handle.
    \end{verse}
    \vskip0.5in
    \begin{verse}
       A model is used, it may lead to incorrect predictions. The model is
       often modified, frequently discarded, and sometimes used anyway because
       it is better than nothing. This is the way science develops.  
    \end{verse}
\end{frame}





