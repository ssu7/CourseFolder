\documentclass[12pt]{article}
\usepackage{termcal}
\usepackage[colorlinks=true,pagebackref,linkcolor=blue]{hyperref}
\textwidth=7in
\textheight=9.5in
\topmargin=-1in
\headheight=0in
\headsep=.5in
\hoffset  -.85in

\pagestyle{empty}

\renewcommand{\thefootnote}{\fnsymbol{footnote}}


\newcommand{\MWClass}{%
\calday[Monday]{\classday} % Monday
\skipday % Tuesday (no class)
\calday[Wednesday]{\classday} % Wednesday
\skipday % Thursday (no class)
\calday[Friday]{}
\skipday\skipday 
}

\newcommand{\TRClass}{%
\skipday % Monday (no class)
\calday[Tuesday]{\classday} % Tuesday
\skipday % Wednesday (no class)
\calday[Thursday]{\classday} % Thursday
\skipday % Friday 
\skipday\skipday % weekend (no class)
}

\newcommand{\Holiday}[2]{%
\options{#1}{\noclassday}
\caltext{#1}{#2}
}


\begin{document}



\begin{center}
{\bf AMS 550.400 \quad MW 3:00 - 4:15 PM \quad  Room: Hodson 211}\\
\vskip.2in
{\footnotesize Last Compiled on \today}
\end{center}

\setlength{\unitlength}{1in}

\begin{picture}(6,.1) 
\put(0,0) {\line(1,0){6.25}}         
\end{picture}

 

\renewcommand{\arraystretch}{2}

\vskip.25in
\noindent\textbf{Instructor:} Dr. N.~H.~Lee (nhlee@jhu.edu) 
\vskip.25in
\noindent\textbf{Office Hours:} To Be Arranged 

\vskip.25in
\noindent\textbf{Course Webpage:} Blackboard will be utilized.  Access it
through the JHU portal:
\begin{center}
    \url{http://my.jhu.edu}
\end{center} 
\vskip.25in
\noindent\textbf{Required Textbook:}  
Either an \emph{electronic} or a \emph{physical} copy is required for each of the following books:
\begin{itemize}
    \item E.\ A.\ Bender, \it An Introduction to Mathematical Modeling, \rm 
    \item J.\ E.\ Miller, \it The Chicago Guide to Writing about Multivariate Analysis. \rm
\end{itemize}
%The following book is also recommended for reference:
%\begin{itemize}
%    \item M.\ Kutner, C.\ Nachstsheim, J.\ Neter and W.\ Li, \it Applied Linear Statistical Models.
%\end{itemize}
\vskip.25in
\noindent\textbf{Prerequisites:} %\footnotemark
Students are expected to have some exposure to:
\begin{itemize}
    \item Probability and Statistics (550.310/550.311 or equivalent),
    \item Multivariate Calculus (110.202 or equivalent).
\end{itemize}
%\footnotetext{Footnote text goes here.}

\vspace*{.15in}

\noindent\textbf{Grade Policy:} 
I expect interest and enthusiasm from the students in this class.  
The class participation is 30\%, which includes homework assignments, 
often for class discussion and your resume. The course project is 70\%, which 
further breaks down as follows:  
a work statement (10\%), 
a midterm presentation (10\%), 
a progress report (10\%),
a final presentation (20\%) 
and a final report (20\%).  
The project must be approved by the instructor prior to your start of writing
your work statement. 
The final presentation slides and the final report are to be typed in \LaTeX\
using the course templates.  The templates are distributed in class.
The final presentation can be delivered in class if you prefer,
but it is also allowed to be delivered by a video-recording as long as you
have the necessary means to do so. 

\vskip.25in
\noindent \textbf{Course Objectives}: 
This course has three objectives: 
\begin{itemize}
    \item To gain experience in mathematical modeling,
    \item To gain experience in presenting technical results,
    \item To be aware of effective technical typesetting tools.
\end{itemize}
For illustration of principles of mathematical modeling, 
anecdotes of successful mathematical models are discussed.  

To facilitate developing a proper style for presenting technical 
results, adopted is a particular style of writing useful for 
summarizing a statistical analysis for multivariate data.
   
For effective typesetting tools, tutorials for Vim, \LaTeX, 
Beamer, PGF/TikZ, R and Git may be given and for enhanced learning
experience, these tools may be used throughout the class.

\vspace*{.15in}
\noindent\textbf{Equipment:}
You will need a computer with Git, \LaTeX\ and R installed. 
If you have a laptop, you should install the most recent version of Git, R  and
\LaTeX\ on it. 
This will be discussed. If you do not have your own laptop, R and \LaTeX\ 
may be available in the Krieger labs and on computers in the AMS Computing Lab
(on which AMS masters' students have priority).  However, relying on the
softwares in the Krieger labs and AMS Computing Lab is unlikely to be
satisfactory. 

\vskip.25in
\noindent\textbf{Academic Honesty}:  
The strength of the university depends on academic and personal integrity. In
this course, you must be honest and truthful. Ethical violations include
cheating on exams, plagiarism, reuse of assignments, improper use of the
Internet and electronic devices, unauthorized collaboration, alteration of
graded assignments, forgery and falsification, lying, facilitating academic
dishonesty, and unfair competition.

In this course, students are permitted and indeed encouraged to discuss
homework problems with one another, but it is expected that the writing up of
answers will be done privately. Copying by one student of another's homework
solutions is considered an ethics violation in this course. Discussion of
lecture material is strongly encouraged.

Report any violations you witness to the instructor. You may consult the
associate dean of students and/or the chairman of the Ethics Board beforehand.
See the guide on ``Academic Ethics for Undergraduates'' the Ethics Board web
site for more information:
\begin{center}
\url{http://ethics.jhu.edu} 
\end{center}

\vskip.25in
\noindent\textbf{University Attendance Policy}:  
Students are expected to attend classes regularly. A student who incurs an excessive
number of absences may be withdrawn from a class at the discretion of the
professor:
\begin{center}
\url{http://www.jhu.edu/design/oliver/academic_manual/illness.html}
\end{center}
In particular, each set of three missed classes is equivalent to 10\% deduction 
in your grade.  For example, if you have missed the total of three classes,
then you can at most get B in your class.  If six classes are missed,
then C becomes the maximum grade possible.  Upon the instructor's approval,
you may be allowed to make up few classes by way of extra assignments such 
as writing book reports. 

\vskip.25in
\noindent\textbf{Important Dates}:
\begin{center} \begin{minipage}{5in} \begin{flushleft} Last day to add \dotfill Fri, Sep 14\\
    Last day to drop \dotfill Sun, Oct 14\\
    Fall Break Day \dotfill Mon, Oct 15\\
    Classes meet according to Monday Schedule\dotfill Tue, Oct 16\\
    Last day to withdraw \dotfill Fri, Oct 26\\
    Thanksgiving vacation\dotfill Wed, Nov 21 -- Sun, Nov 25\\
    Reading period\dotfill Sat, Dec 8 -- Tue, Dec 11\\
\end{flushleft}
\end{minipage}
\end{center}

\newpage

\paragraph*{Tentative Schedule:}
\hoffset  0in
\begin{center}
\begin{calendar}{09/03/2012}{14} 
    \centering
\setlength{\calboxdepth}{.3in}
\MWClass
% schedule
\Holiday{9/3/2012}{\emph{Labor Day}}
\Holiday{9/14/2012}{\emph{Due: Resume}}
\Holiday{9/28/2012}{\emph{Due: Work Statement}}
\Holiday{10/15/2012}{\emph{N.B. Fall Break} \\ Choosing Effective Examples and Analogies}    
\Holiday{10/12/2012}{\emph{Due: Midterm Presentation}}
\Holiday{10/26/2012}{\emph{Due: Progress Report}}
\Holiday{11/21/2012}{\emph{Thanksgiving Break Begins}}
\Holiday{11/16/2012}{\emph{Due: Final Presentation}}
\Holiday{11/30/2012}{\emph{Due: Final Report}}
%\Holiday{12/7/2012}{\emph{Due: Revised Final Report}}
\caltexton{1}{Overview}
%\caltextnext{Insurance Redlining}    
%\caltextnext{Insurance Redlining}    
\caltextnext{Seven Basic Principles}
\caltextnext{Seven Basic Principles}
\caltextnext{Causality, Statistical Significance, and Substantive Significance}
\caltextnext{Causality, Statistical Significance, and Substantive Significance}
\caltextnext{Five More Technical Principles}
\caltextnext{Five More Technical Principles}
\caltextnext{Creating Effective Table} 
\caltextnext{Creating Effective Table}  
\caltextnext{Creating Effective Chart}  
\caltextnext{Creating Effective Chart} 
%\caltextnext{Basic Optimization\quad (N.B.\ \emph{Fall Break, i.e., no class on Monday, but we meet instead
%on Tuesday, Oct 16 from 3.00pm to 4.15pm})} 
%\caltextnext{Argument From Scale}    
\caltextnext{Choosing Effective Examples and Analogies}      
\caltextnext{Basic Types of Quantitative Comparisons}       
\caltextnext{Basic Types of Quantitative Comparisons}       
\caltextnext{Quantitative Comparisons for Multivariate Models} 
\caltextnext{Quantitative Comparisons for Multivariate Models} 
\caltextnext{Choosing How to Present Statistical Test Results}
\caltextnext{Choosing How to Present Statistical Test Results}
\caltextnext{Arguments from Scale}     
\caltextnext{Arguments from Scale}     
\caltextnext{Graphical Methods}      
\caltexton{22}{Graphical Methods}     
\caltextnext{Basic Optimization}   
\caltextnext{Basic Optimization}   
\caltextnext{Basic Optimization}   
% ... and so on
\end{calendar}
\end{center}
\end{document}

\caltextnext

\caltextnext{Writing about Multivariate Models}
