\section{Project}

\begin{frame}
    \frametitle{Outline}
    \tableofcontents[currentsection, currentsubsection]
\end{frame}


\begin{frame}[fragile]
    \frametitle{Mission Impossible?: an analogy}
    \begin{figure}
        \caption{Mission Impossible Season 2 Episode (00:00 -- 06:25)}
    \begin{center}
    \href{http://movies.netflix.com/WiPlayer?movieid=70157337&trkid=4431095&pt_request_id=a8a7108c-c068-4d22-b649-0b7095279045-1882594&pt_rank=4&pt_row=-1&pt_location=WATCHNOW#MovieId=70157337&EpisodeMovieId=70156671}{
            \includegraphics[width=0.8\textwidth]{images/IMFproblemstatement.png}
    }
    \end{center}
    \end{figure}
\end{frame}

\begin{frame}
    \frametitle{Project in Industry: Frequently Recurring Elements}
    A stylized timeline:
    \vspace{7pt}
             \begin{enumerate}
                 \item Work Statement,
                 \item Midterm Presentation,
                 \item Progress Report,
                 \item Final Presentation,
                 \item Final Report.
             \end{enumerate}
    \begin{center}
        \href{http://www.ipam.ucla.edu/programs/rips2011/}{
        \includegraphics[width=0.5\textwidth]{images/ipam}}        
    \end{center}
\end{frame}

\subsection{Work Statement}

\begin{frame}
    \frametitle{Outline}
    \tableofcontents[currentsection, currentsubsection]
\end{frame}

\begin{frame}
    \frametitle{What is Work Statement}
This is the written proposal and definition of the project and constitutes the
team's ``contract'' with the sponsor. It should be approximately 2-5 pages
long.  It sets forth the nature of the project, the specific
objectives of the project, the results expected, and the ``deliverables'' for
the project. The scope of the project must be within the timetable for the
program and that the deliverables are reasonable and appropriate; given the
nature of research, it should not include promises that the team cannot be
certain to achieve. It is ultimately given to the sponsor for review and
signature.
\end{frame}


\begin{frame}
    \frametitle{Template 1}
    \begin{enumerate}
        \item Abstract
        \item Background
        \item Problem description
        \item Approach (``time permitting'' clause for some work)
        \item Schedule (dates for completing milestones and tasks and for
            deliverables)
        \item Milestones (major checkpoints your team will use to stay on
            track)
        \item Deliverables (specific work products you will deliver to the
            sponsor)
    \end{enumerate}
\end{frame}

\begin{frame}
    \frametitle{Templates 2}
    \begin{enumerate}
        \item Introduction
        \item Problem background
        \item Mathematical background
        \item Computing background 
        \item Possible solutions and project objectives
        \item Deliverables (``time permitting'' clause for some work)
        \item Timeline
    \end{enumerate}
\end{frame}

\begin{frame}
    \frametitle{Template 3}
    \begin{enumerate}
        \item Project background
        \item Goals (major direction you see the work aimed at, not
            necessarily what you bid to do)
        \item Proposed mathematical approach
        \item Objectives (specific aims of your project, and schedule of
            results you expect to achieve)
        \item Optional objectives 
        \item Deliverables
        \item Milestones
        \item Work flowchart
        \item Schedule
    \end{enumerate}
\end{frame}

\begin{frame}
    \frametitle{Template 4}
    \begin{enumerate}
        \item Abstract
        \item Problem background
        \item Problem description
        \item Approach 
        \item Deliverables
        \item Timetable
        \item Team members
    \end{enumerate}
\end{frame}

\begin{frame}
    \frametitle{Work Statement}
    \begin{block}
        {In the initial segment (``Abstract'', ``Introduction'', ``Background'')}
        \begin{itemize}
            \item Brief description of the company
            \item Major product lines(s)
            \item A brief (abstract) description of the project
        \end{itemize}
    \end{block}
\end{frame}

\begin{frame}
    \frametitle{Work Statement}
    \begin{block}
        {Throughout}
        \begin{itemize}
            \item Spell out terminology -- avoid undefined jargon or acronym
            \item When options must be resolved, give dates by which they must
                be resolved
            \item Give modest objectives, not boastful ones
        \end{itemize}
    \end{block}
\end{frame}

\begin{frame}
    \frametitle{Work Statement}
    \begin{block}
        {List of deliverables should include}
        \begin{itemize}
            \item Site visits (to be arranged)
            \item Midterm oral presentation
            \item Midterm report
            \item Final presentation
            \item Final report
            \item Software (if appropriate) 
                \begin{itemize}
                    \item Specify sponsor-approved OS, platform
                    \item Documentations
                \end{itemize}
        \end{itemize}
    \end{block}
\end{frame}

\begin{frame}
    \frametitle{Work Statement}
    \begin{center}
        \href{http://www.ipam.ucla.edu/programs/rips2002/rips2002_projects.html}{Work Statement Examples}
    \end{center}
    \begin{description}
        \item[\quad] See \emph{Protein Pathways} Project Work Statement 
    \end{description}
\end{frame}

\subsection{Glossary}

\begin{frame}[allowframebreaks]
    \frametitle{Glossary}
    \begin{block}
        {GOAL} The overall, long range, end result that your research is aimed at, what
you are trying to achieve ultimately. Stating a goal does not mean you believe
you will get there this time around. It is the grand view towards which you
strive. The goal of AIDS research is to find a cure for AIDS.
\end{block}
\end{frame}

\begin{frame}
    \frametitle{Glossary}
    \begin{block}
        {OBJECTIVES} The specific things you will try to achieve in your project, the
immediate targets of your research. Your objectives spell out how you have
parsed the problem of heading towards the goal into smaller pieces that you
will work on. The objectives set practical limits on your work. They point to
where the project can reasonably expect to wind up. It should be clear that
the objectives fit into and work towards the long-range goal.
\end{block}
\end{frame}

\begin{frame}
    \frametitle{Glossary}
\begin{block}
        {TASKS} These are the specific things you will do in order to achieve your
objectives. The tasks drive your determination of what skills and other
resources (such as data, software, hardware, written materials, work
environment) will be needed for your project. If among the resources needed
are ones that must be supplied by the sponsor, then you will need to specify
these items in your Work Statement.
\end{block}
    
\end{frame}


\begin{frame}
    \frametitle{Glossary}
\begin{block}
        {DELIVERABLES} The things you promise to deliver to the sponsor. For a 
project, these include a mid-term and final report, a mid-term presentation
and a final presentation on Projects Day. They may also include site visits to
the sponsor (usually one near the beginning of the project to get acquainted
with the sponsor, and one after Projects Day to present the work at the
sponsor's location), software, perhaps hardware in some cases, written results
of literature searches, white papers (i.e., written background information on
such things as plans, methods or concepts prepared for internal use), etc.
These additional items are to be decided by you in consultation with your
sponsor’s mentor.
\end{block}
    
\end{frame}

\begin{frame}
    \frametitle{Glossary}
\begin{block}
        {MILESTONES} A list of specific accomplishments that you may use to mark
progress and maintain pace and coordination within your project. They are used
to help your team stay on track and to determine the success of a chosen line
of attack on your problem. Milestones may or may not be included in your Work
Statement, but you should definitely think these through for your own use as
you plan your project and Work Statement. They are check-points for you (and
for your sponsor, if they are included in the Work Statement), not necessarily
deliverables. You may want to specify major milestones in your Work Statements
to indicate what you would do if your research leads 
to the conclusion that some objective cannot be accomplished. For example, "if
by such a date we have found it impossible to achieve X, then we will begin
Y." Research is exploration of the unknown, so you may encounter an
intractable obstacle and need to work around it. You can't know everything
ahead of time. Give some thought to this and try to allow for milestones by
which you can judge where you are and what you need to do to proceed
effectively in the event you don't meet a milestone.
\end{block}
    
\end{frame}

\begin{frame}
    \frametitle{Glossary}
\begin{block}
        {SCHEDULE} This specifies when you will finish major parts of your research and
provides a timetable for completion of deliverables. Internally, you should
maintain as fine-grained a schedule as you need to keep your team coordinated
and on track, but in your Work Statement it is best to make the schedule and
list of deliverables as modest as the sponsor will allow.
\end{block}
    
\end{frame}
